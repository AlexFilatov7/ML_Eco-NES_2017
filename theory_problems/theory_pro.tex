\documentclass[11pt, a4paper]{article}

% If you can't see cyrillic letters in R-studio choose
% File-Reopen with encoding
% utf8 is the preferred encoding


%%%%%%%%%%%%%%%%%%%%%%%  Загрузка пакетов  %%%%%%%%%%%%%%%%%%%%%%%%%%%%%%%%%%
% кусок от урсса
%\usepackage[60x90,headers,11pt]{format}

%\textheight=494pt%
%\textwidth=322pt%
%
%\oddsidemargin=0pt%
%\evensidemargin=0pt
%\topmargin=-1pt \headsep=14pt \headheight=22pt \voffset=-28pt
%\hoffset=-50pt


\clubpenalty=10000
\widowpenalty=10000

%\overfullrule=5pt
%\hfuzz=1.5mm
%\baselineskip=12pt plus 0.18pt minus 0.1pt


%\pagestyle{headings}
% конец куска от урсса




% специальная версия для knitr'а. Исключает graphicx

%\usepackage{showkeys} % показывать метки в готовом pdf

\usepackage{etex} % расширение классического tex
% в частности позволяет подгружать гораздо больше пакетов, чем мы и займёмся далее

%\usepackage{mathtext} % русские буквы в формулах? (и без неё работает)
% Например, $x_{\text{один}}$

%\usepackage{cmap} % для поиска русских слов в pdf --- теперь работает без этого
% а с cmap не работает печать на принтер ;)
\usepackage{verbatim} % для многострочных комментариев
\usepackage{makeidx} % для создания предметных указателей
\usepackage[X2, T2A]{fontenc}
\usepackage[utf8]{inputenc} % задание utf8 кодировки исходного tex файла
\usepackage{setspace}
\usepackage{amsmath, amsfonts, amssymb, amsthm}
\usepackage{mathrsfs} % sudo yum install texlive-rsfs
\usepackage{dsfont} % sudo yum install texlive-doublestroke
\usepackage{array, multicol, multirow, bigstrut} % sudo yum install texlive-multirow
\usepackage{indentfirst} % установка отступа в первом абзаце главы
\usepackage[russian]{babel} % выбор языка для документа
\usepackage{bm}
\usepackage{bbm} % шрифт с двойными буквами
%\usepackage[perpage]{footmisc}

\usepackage{dcolumn} % центрирование по разделителю для apsrtable

% создание гиперссылок в pdf
\usepackage[pdftex, unicode, colorlinks=true, urlcolor=blue, hyperindex, breaklinks]{hyperref}

% свешиваем пунктуацию
% теперь знаки пунктуации могут вылезать за правую границу текста, при этом текст выглядит ровнее
\usepackage{microtype}

\usepackage{textcomp}  % Чтобы в формулах можно было русские буквы писать через \text{}

% размер листа бумаги
%\usepackage[paperwidth=145mm,paperheight=215mm,
%height=182mm,width=113mm,top=20mm,includefoot]%{geometry}
\usepackage[paper=a4paper, top=15mm, bottom=13.5mm, left=16.5mm, right=13.5mm, includefoot]{geometry}

\usepackage{xcolor}

% \usepackage[pdftex]{graphicx} % для вставки графики, убрано, т.к. knitr похоже сам добавляет

\usepackage{float, longtable}
\usepackage{soulutf8}

\usepackage{enumitem} % дополнительные плюшки для списков
%  например \begin{enumerate}[resume] позволяет продолжить нумерацию в новом списке

\usepackage{mathtools}
\usepackage{cancel,xspace} % sudo yum install texlive-cancel

% \usepackage{minted} % display program code with syntax highlighting
% требует установки pygments и python

\usepackage{numprint} % sudo yum install texlive-numprint
\npthousandsep{,}\npthousandthpartsep{}\npdecimalsign{.}

\usepackage{embedfile} % Чтобы код LaTeXа включился как приложение в PDF-файл

\usepackage{subfigure} % для создания нескольких рисунков внутри одного

\usepackage{tikz, pgfplots} % язык для рисования графики из latex'a
\usetikzlibrary{trees} % tikz-прибамбас для рисовки деревьев
\usepackage{tikz-qtree} % альтернативный tikz-прибамбас для рисовки деревьев
\usetikzlibrary{arrows} % tikz-прибамбас для рисовки стрелочек подлиннее

\usepackage{todonotes} % для вставки в документ заметок о том, что осталось сделать
% \todo{Здесь надо коэффициенты исправить}
% \missingfigure{Здесь будет Последний день Помпеи}
% \listoftodos --- печатает все поставленные \todo'шки


% более красивые таблицы
\usepackage{booktabs}
% заповеди из докупентации:
% 1. Не используйте вертикальные линни
% 2. Не используйте двойные линии
% 3. Единицы измерения - в шапку таблицы
% 4. Не сокращайте .1 вместо 0.1
% 5. Повторяющееся значение повторяйте, а не говорите "то же"



%\usepackage{asymptote} % пакет для рисовки графики, должен идти после graphics
% но мы переходим на tikz :)

%\usepackage{sagetex} % для интеграции с Sage (вероятно тоже должен идти после graphics)

% metapost создает упрощенные eps файлы, которые можно напрямую включать в pdf
% эта группа команд декларирует, что файлы будут этого упрощенного формата
% если metapost не используется, то этот блок не нужен
\usepackage{ifpdf} % для определения, запускается ли pdflatex или просто латех
\ifpdf
	\DeclareGraphicsRule{*}{mps}{*}{}
\fi
%%%%%%%%%%%%%%%%%%%%%%%%%%%%%%%%%%%%%%%%%%%%%%%%%%%%%%%%%%%%%%%%%%%%%%


%%%%%%%%%%%%%%%%%%%%%%%  Внедрение tex исходников в pdf файл  %%%%%%%%%%%%%%%%%%%%%%%%%%%%%%%%%%
\embedfile[desc={Main tex file}]{\jobname.tex} % Включение кода в выходной файл
\embedfile[desc={title_bor}]{title_bor_utf8_knitr.tex}

%%%%%%%%%%%%%%%%%%%%%%%%%%%%%%%%%%%%%%%%%%%%%%%%%%%%%%%%%%%%%%%%%%%%%%



%%%%%%%%%%%%%%%%%%%%%%%  ПАРАМЕТРЫ  %%%%%%%%%%%%%%%%%%%%%%%%%%%%%%%%%%
\setstretch{1}                          % Межстрочный интервал
\flushbottom                            % Эта команда заставляет LaTeX чуть растягивать строки, чтобы получить идеально прямоугольную страницу
\righthyphenmin=2                       % Разрешение переноса двух и более символов
%\pagestyle{plain}                       % Нумерация страниц снизу по центру.
%\widowpenalty=300                     % Небольшое наказание за вдовствующую строку (одна строка абзаца на этой странице, остальное --- на следующей)
%\clubpenalty=3000                     % Приличное наказание за сиротствующую строку (омерзительно висящая одинокая строка в начале страницы)
\setlength{\parindent}{1.5em}           % Красная строка.
%\captiondelim{. }
\setlength{\topsep}{0pt}
%%%%%%%%%%%%%%%%%%%%%%%%%%%%%%%%%%%%%%%%%%%%%%%%%%%%%%%%%%%%%%%%%%%%%%



%%%%%%%% Это окружение, которое выравнивает по центру без отступа, как у простого center
\newenvironment{center*}{%
  \setlength\topsep{0pt}
  \setlength\parskip{0pt}
  \begin{center}
}{%
  \end{center}
}
%%%%%%%%%%%%%%%%%%%%%%%%%%%%%%%%%%%%%%%%%%%%%%%%%%%%%%%%%%%%%%%%%%%%%%


%%%%%%%%%%%%%%%%%%%%%%%%%%% Правила переноса  слов
\hyphenation{ }
%%%%%%%%%%%%%%%%%%%%%%%%%%%%%%%%%%%%%%%%%%%%%%%%%%%%%%%%%%%%%%%%%%%%%%

\emergencystretch=2em


% DEFS
\def \mbf{\mathbf}
\def \msf{\mathsf}
\def \mbb{\mathbb}
\def \tbf{\textbf}
\def \tsf{\textsf}
\def \ttt{\texttt}
\def \tbb{\textbb}

\def \wh{\widehat}
\def \wt{\widetilde}
\def \ni{\noindent}
\def \ol{\overline}
\def \cd{\cdot}
\def \fr{\frac}
\def \bs{\backslash}
\def \lims{\limits}
\DeclareMathOperator{\dist}{dist}
\DeclareMathOperator{\VC}{VCdim}
\DeclareMathOperator{\card}{card}
\DeclareMathOperator{\sign}{sign}
\DeclareMathOperator{\sgn}{sign}
\DeclareMathOperator{\Tr}{\mbf{Tr}}
\DeclareMathOperator{\tr}{tr}


\def \xfs{(x_1,\ldots,x_{n-1})}
\DeclareMathOperator*{\argmin}{arg\,min}
\DeclareMathOperator*{\amn}{arg\,min}
\DeclareMathOperator*{\amx}{arg\,max}
\DeclareMathOperator{\trace}{tr}
\DeclareMathOperator{\rk}{rk}


\DeclareMathOperator{\Corr}{Corr}
\DeclareMathOperator{\sCorr}{sCorr}
\DeclareMathOperator{\sCov}{sCov}
\DeclareMathOperator{\sVar}{sVar}

\DeclareMathOperator{\Cov}{Cov}
\DeclareMathOperator{\Var}{Var}
\DeclareMathOperator{\corr}{Corr}
\DeclareMathOperator{\cov}{Cov}
\DeclareMathOperator{\var}{Var}
\DeclareMathOperator{\bin}{Bin}
\DeclareMathOperator{\Bin}{Bin}
\DeclareMathOperator{\rang}{rang}
\DeclareMathOperator*{\plim}{plim}
\DeclareMathOperator{\MSE}{MSE}

\providecommand{\iff}{\Leftrightarrow}
\providecommand{\hence}{\Rightarrow}

\def \ti{\tilde}
\def \wti{\widetilde}

\def \mA{\mathcal{A}}
\def \mB{\mathcal{B}}
\def \mC{\mathcal{C}}
\def \mE{\mathcal{E}}
\def \mF{\mathcal{F}}
\def \mH{\mathcal{H}}
\def \mL{\mathcal{L}}
\def \mN{\mathcal{N}}
\def \mU{\mathcal{U}}
\def \mV{\mathcal{V}}
\def \mW{\mathcal{W}}


\def \R{\mbb R}
\def \N{\mbb N}
\def \Z{\mbb Z}
\def \P{\mbb{P}}
\def \p{\mbb{P}}
\newcommand{\E}{\mathbb{E}}
\def \D{\msf{D}}
\def \I{\mbf{I}}

\def \QQ{\mbb Q}
\def \RR{\mbb R}
\def \NN{\mbb N}
\def \ZZ{\mbb Z}
\def \PP{\mbb P}


\def \a{\alpha}
\def \b{\beta}
\def \t{\tau}
\def \dt{\delta}
\newcommand{\e}{\varepsilon}
\def \ga{\gamma}
\def \kp{\varkappa}
\def \la{\lambda}
\def \sg{\sigma}
\def \sgm{\sigma}
\def \tt{\theta}
\def \ve{\varepsilon}
\def \Dt{\Delta}
\def \La{\Lambda}
\def \Sgm{\Sigma}
\def \Sg{\Sigma}
\def \Tt{\Theta}
\def \Om{\Omega}
\def \om{\omega}

%\newcommand{\p}{\partial}

\def \ni{\noindent}
\def \lq{\glqq}
\def \rq{\grqq}
\def \lbr{\linebreak}
\def \vsi{\vspace{0.1cm}}
\def \vsii{\vspace{0.2cm}}
\def \vsiii{\vspace{0.3cm}}
\def \vsiv{\vspace{0.4cm}}
\def \vsv{\vspace{0.5cm}}
\def \vsvi{\vspace{0.6cm}}
\def \vsvii{\vspace{0.7cm}}
\def \vsviii{\vspace{0.8cm}}
\def \vsix{\vspace{0.9cm}}
\def \VSI{\vspace{1cm}}
\def \VSII{\vspace{2cm}}
\def \VSIII{\vspace{3cm}}

\newcommand{\bls}[1]{\boldsymbol{#1}}
\newcommand{\bsA}{\boldsymbol{A}}
\newcommand{\bsH}{\boldsymbol{H}}
\newcommand{\bsI}{\boldsymbol{I}}
\newcommand{\bsP}{\boldsymbol{P}}
\newcommand{\bsR}{\boldsymbol{R}}
\newcommand{\bsS}{\boldsymbol{S}}
\newcommand{\bsX}{\boldsymbol{X}}
\newcommand{\bsY}{\boldsymbol{Y}}
\newcommand{\bsZ}{\boldsymbol{Z}}
\newcommand{\bse}{\boldsymbol{e}}
\newcommand{\bsq}{\boldsymbol{q}}
\newcommand{\bsy}{\boldsymbol{y}}
\newcommand{\bsbeta}{\boldsymbol{\beta}}
\newcommand{\fish}{\mathrm{F}}
\newcommand{\Fish}{\mathrm{F}}
\renewcommand{\phi}{\varphi}
\newcommand{\ind}{\mathds{1}}
\newcommand{\inds}[1]{\mathds{1}_{\{#1\}}}
\renewcommand{\to}{\rightarrow}
\newcommand{\sumin}{\sum\limits_{i=1}^n}
\newcommand{\ofbr}[1]{\bigl( \{ #1 \} \bigr)}     % Например, вероятность события. Большие круглые, нормальные фигурные скобки вокруг аргумента
\newcommand{\Ofbr}[1]{\Bigl( \bigl\{ #1 \bigr\} \Bigr)} % Например, вероятность события. Больше больших круглые, большие фигурные скобки вокруг аргумента
\newcommand{\oeq}{{}\textcircled{\raisebox{-0.4pt}{{}={}}}{}} % Равно в кружке
\newcommand{\og}{\textcircled{\raisebox{-0.4pt}{>}}}  % Знак больше в кружке

% вместо горизонтальной делаем косую черточку в нестрогих неравенствах
\renewcommand{\le}{\leqslant}
\renewcommand{\ge}{\geqslant}
\renewcommand{\leq}{\leqslant}
\renewcommand{\geq}{\geqslant}


\newcommand{\figb}[1]{\bigl\{ #1  \bigr\}} % большие фигурные скобки вокруг аргумента
\newcommand{\figB}[1]{\Bigl\{ #1  \Bigr\}} % Больше больших фигурные скобки вокруг аргумента
\newcommand{\parb}[1]{\bigl( #1  \bigr)}   % большие скобки вокруг аргумента
\newcommand{\parB}[1]{\Bigl( #1  \Bigr)}   % Больше больших круглые скобки вокруг аргумента
\newcommand{\parbb}[1]{\biggl( #1  \biggr)} % большие-большие круглые скобки вокруг аргумента
\newcommand{\br}[1]{\left( #1  \right)}    % круглые скобки, подгоняемые по размеру аргумента
\newcommand{\fbr}[1]{\left\{ #1  \right\}} % фигурные скобки, подгоняемые по размеру аргумента
\newcommand{\eqdef}{\mathrel{\stackrel{\rm def}=}} % знак равно по определению
\newcommand{\const}{\mathrm{const}}        % const прямым начертанием
\newcommand{\zdt}[1]{\textit{#1}}
\newcommand{\ENG}[1]{\foreignlanguage{british}{#1}}
\newcommand{\ENGs}{\selectlanguage{british}}
\newcommand{\RUSs}{\selectlanguage{russian}}
\newcommand{\iid}{\text{i.\hspace{1pt}i.\hspace{1pt}d.}}

\newdimen\theoremskip
\theoremskip=0pt
\newenvironment{note}{\par\vskip\theoremskip\textbf{Замечание.\xspace}}{\par\vskip\theoremskip}
\newenvironment{hint}{\par\vskip\theoremskip\textbf{Подсказка.\xspace}}{\par\vskip\theoremskip}
\newenvironment{ist}{\par\vskip\theoremskip Источник:\xspace}{\par\vskip\theoremskip}

\newcommand*{\tabvrulel}[1]{\multicolumn{1}{|c}{#1}}
\newcommand*{\tabvruler}[1]{\multicolumn{1}{c|}{#1}}

\newcommand{\II}{{\fontencoding{X2}\selectfont\CYRII}}   % I десятеричное (английская i неуместна)
\newcommand{\ii}{{\fontencoding{X2}\selectfont\cyrii}}   % i десятеричное
\newcommand{\EE}{{\fontencoding{X2}\selectfont\CYRYAT}}  % ЯТЬ
\newcommand{\ee}{{\fontencoding{X2}\selectfont\cyryat}}  % ять
\newcommand{\FF}{{\fontencoding{X2}\selectfont\CYROTLD}} % ФИТА
\newcommand{\ff}{{\fontencoding{X2}\selectfont\cyrotld}} % фита
\newcommand{\YY}{{\fontencoding{X2}\selectfont\CYRIZH}}  % ИЖИЦА
\newcommand{\yy}{{\fontencoding{X2}\selectfont\cyrizh}}  % ижица

%%%%%%%%%%%%%%%%%%%%% Определение разрядки разреженного текста и задание красивых многоточий
\sodef\so{}{.15em}{1em plus1em}{.3em plus.05em minus.05em}
\newcommand{\ldotst}{\so{...}}
\newcommand{\ldotsq}{\so{?\hbox{\hspace{-0.61ex}}..}}
\newcommand{\ldotse}{\so{!..}}
%%%%%%%%%%%%%%%%%%%%%%%%%%%%%%%%%%%%%%%%%%%%%%%%%%%%%%%%%%%%%%%%%%%%%%

%%%%%%%%%%%%%%%%%%%%%%%%%%%%% Команда для переноса символов бинарных операций
\def\hm#1{#1\nobreak\discretionary{}{\hbox{$#1$}}{}}
%%%%%%%%%%%%%%%%%%%%%%%%%%%%%%%%%%%%%%%%%%%%%%%%%%%%%%%%%%%%%%%%%%%%%%

%\setlist[enumerate,1]{label=\arabic*., ref=\arabic*, partopsep=0pt plus 2pt, topsep=0pt plus 1.5pt,itemsep=0pt plus .5pt,parsep=0pt plus .5pt}
%\setlist[itemize,1]{partopsep=0pt plus 2pt, topsep=0pt plus 1.5pt,itemsep=0pt plus .5pt,parsep=0pt plus .5pt}

% Эти парни затем, если вдруг не захочется управлять списками из-под уютненького enumitem
% или если будет жизненно важно, чтобы в списках были именно русские буквы.
%\setlength{\partopsep}{0pt plus 3pt}
%\setlength{\topsep}{0pt plus 2pt}
%\setlength{\itemsep}{0 plus 1pt}
%\setlength{\parsep}{0 plus 1pt}

%на всякий случай пока есть
%теоремы без нумерации и имени
%\newtheorem*{theor}{Теорема}

%"Определения","Замечания"
%и "Гипотезы" не нумеруются
%\newtheorem*{defin}{Определение}
%\newtheorem*{rem}{Замечание}
%\newtheorem*{conj}{Гипотеза}

%"Теоремы" и "Леммы" нумеруются
%по главам и согласованно м/у собой
%\newtheorem{theorem}{Теорема}
%\newtheorem{lemma}[theorem]{Лемма}

% Утверждения нумеруются по главам
% независимо от Лемм и Теорем
%\newtheorem{prop}{Утверждение}
%\newtheorem{cor}{Следствие}


\def \useR{$[$R$]$ }

%% эконометрические сокращения
\def \hb{\hat{\beta}}
\def \hs{\hat{s}}
\def \hy{\hat{y}}
\def \hY{\hat{Y}}
\def \he{\hat{\varepsilon}}
\def \v1{\vec{1}}
\def \e{\varepsilon}
\def \z{z}
\def \hVar{\widehat{\Var}}
\def \hCorr{\widehat{\Corr}}
\def \hCov{\widehat{\Cov}}


%% лаг
\renewcommand{\L}{\mathrm{L}}





\usepackage[bibencoding = auto, backend = biber,
sorting = none]{biblatex}

\addbibresource{metrics_pro.bib}

\def \RR{\mathbb{R}}
\def \cN{\mathcal{N}}
\def \htheta{\hat{\theta}}

\title{Задачки для семинаров и домашки}
\author{Винни-Пух}
\date{\today}


% делаем короче интервал в списках
\setlength{\itemsep}{0pt}
\setlength{\parskip}{0pt}
\setlength{\parsep}{0pt}


\DeclareMathOperator{\Med}{Med}


\usepackage{answers} % дележка условий и ответов

%\newtheorem{problem}{Задача}
%\numberwithin{problem}{section}

\Newassociation{sol}{solution}{solution_file}
% sol — имя окружения внутри задач
% solution — имя окружения внутри solution_file
% solution_file — имя файла в который будет идти запись решений
% можно изменить далее по ходу
\Opensolutionfile{solution_file}[all_solutions]
% в квадратных скобках фактическое имя файла


% магия для автоматических гиперссылок задача-решение
\newlist{myenum}{enumerate}{3}
% \newcounter{problem}[chapter] % нумерация задач внутри глав
\newcounter{problem}

\newenvironment{problem}%
{%
\refstepcounter{problem}%
%  hyperlink to solution
     \hypertarget{problem:{\thesection.\theproblem}}{} % нумерация внутри глав
     % \hypertarget{problem:{\theproblem}}{}
     \Writetofile{solution_file}{\protect\hypertarget{soln:\thesection.\theproblem}{}}
     %\Writetofile{solution_file}{\protect\hypertarget{soln:\theproblem}{}}
     \begin{myenum}[label=\bfseries\protect\hyperlink{soln:\thesection.\theproblem}{\thesection.\theproblem},ref=\thesection.\theproblem]
     % \begin{myenum}[label=\bfseries\protect\hyperlink{soln:\theproblem}{\theproblem},ref=\theproblem]
     \item%
    }%
    {%
    \end{myenum}}
% для гиперссылок обратно надо переопределять окружение
% это происходит непосредственно перед подключением файла с решениями





\begin{document}

% \maketitle % ставим сюда название, автора и время создания

\section{Линейная регрессия}

\begin{problem}
Рассмотрим задачу линейной регресии
\[
Q(w) = (y - Xw)^T(y - Xw) \to \min_{w}.
\]

\begin{enumerate}
\item Найдите $dQ(w)$ и $d^2Q(w)$.
\item Выведите формулу для оптимального $w$.
\item Выведите формулу для матрицы-шляпницы (hat-matrix), связывающей вектор фактических $y$ и вектор прогнозов $\hat y = H\cdot y$.
\end{enumerate}

\begin{sol}
\end{sol}
\end{problem}

\begin{problem}
Рассмотрим задачу регрессии с одним признаком и без константы, $\hy_i = w \cdot x_i$. Решите в явном виде задачи МНК со штрафом:

\begin{enumerate}
\item $Q(w) = (y - \hy)^T (y - \hy) + \lambda w^2$;
\item $Q(w) = (y - \hy)^T (y - \hy) + \lambda |w|$;
\end{enumerate}

\begin{sol}
\end{sol}
\end{problem}

\begin{problem}
Храбрая и торопливая исследовательница Мишель хочет решить задачу линейной регрессии по $n$ наблюдениям с вектором $y$ и матрицей признаков $X$. Сначала исследовательница Мишель так торопилась, что совсем забыла последнее наблюдение и оценила задачу с более коротким вектором $y^{-}$ и матрицей $X^{-}$, где не хватает последней строки. Затем Мишель взяла правильную матрицу $X$, но неправильный вектор $y^*$, в котором она вместо фактического последнего наблюдения вектора $y$ вписала его прогноз, полученный с помощью регрессии с $y^{-1}$ и $X^{-}$.

\begin{enumerate}
\item Как связаны $\hat y_n^{-}$ и $\hat y_n^{*}$ (прогнозы для последнего наблюдения полученные по модели без последнего наблюдения и модели с неверным последним наблюдением)?
\item Как выглядит вектор, равный разнице $y - y^*$?
\item Какие величины находятся в векторе $H\cdot (y - y^*)$? Чему равна последняя, $n$-ая, компонента этого вектора? Выразите её через $H_{nn}$ и ошибку прогноза последнего наблюдения по модели без последнего наблюдения, $y_n - \hat y_n^{-}$.
\item Как связаны между собой ошибка прогноза $n$-го наблюдения по полной модели, ошибка прогноза $n$-го наблюдения по модели без последнего наблюдения и $H_{nn}$?
\item Как быстро провести кросс-валидацию с выкидыванием одного наблюдения для задачи линейной регрессии?
\end{enumerate}

\begin{sol}
\[
y_n - \hat y_n = (1 - H_{nn}) (y_n - \hat y_n^-)
\]
\end{sol}
\end{problem}


\section{Линейные классификаторы}

\begin{problem}
Рассмотрим плоскость в $\RR^3$, задаваемую уравнением $5x_1 + 6x_2 -7x_3 + 10 = 0$ и две точки, $A = (2, 1, 4)$ и $B = (4, 0, 4)$.
\begin{enumerate}
  \item Найдите любой вектор, перпендикулярный плоскости.
  \item Правда ли, что отрезок $AB$ пересекает плоскость?
  \item Найдите длину отрезка $AB$;
  \item Не находя расстояние от точек до плоскости, определите, во сколько раз точка $A$ дальше от плоскости, чем точка $B$;
  \item Найдите расстояние от точки $A$ до плоскости.
\end{enumerate}

\begin{sol}
$(5, 6, -7)$
\end{sol}
\end{problem}


\begin{problem}
Рассмотрим простейший персептрон с константой, единственным входом $x_1$ и пороговой функцией активации. Подберите веса так, чтобы персептрон реализовывал логическое отрицание (в ответ на 0 выдавал 1, и наоборот).
\begin{sol}
\end{sol}
\end{problem}

\begin{problem}
Рассмотрим простейший персептрон с константой, двумя входами $x_1$, $x_2$ и пороговой функцией активации.

\todo[inline]{Здесь ассистенты нарисуют в tikz картинку, достойную стоять вместо Джоконды в Лувре}

\begin{enumerate}
\item Подберите веса так, чтобы персептрон реализовывал логическое ИЛИ (OR).
\item Подберите веса так, чтобы персептрон реализовывал логическое И (AND).
\item Докажите, что веса невозможно подобрать так, чтобы персептрон реализовывал исключающее логическое ИЛИ (XOR).
\item Добавьте персептрону вход $x_3 = x_1 \cdot x_2$ и подберите веса так, чтобы персептрон реализовывал XOR.
\item Реализуйте XOR с помощью трёх персептронов с двумя входами и константой. Укажите веса и схему их взаимосвязей.
\end{enumerate}

\begin{sol}
\end{sol}
\end{problem}



\begin{problem}
В коробке завалялось три персептрона, у каждого два входа с константой и пороговая функция активации. Реализуйте с их помощью функцию
\[
y = \begin{cases}
1, \text{ если } x_2 \geq |x_1 - 3| + 2; \\
0, \text{ иначе}
\end{cases}.
\]
\begin{sol}
\end{sol}
\end{problem}



\begin{problem}
Рассмотрим следующий набор данных:

\begin{tabular}{ccc}
\toprule
$x_i$ & $z_i$ & $y_i$ \\
\midrule
-1 & -1 & 0 \\
1 & -1 & 0 \\
-1 & 1 & 0 \\
1 & 1 & 0 \\
0 & 2 & 1 \\
2 & 0 & 1 \\
0 & -2 & 1 \\
-2 & 0 & 1 \\
\bottomrule
\end{tabular}

\begin{enumerate}
\item Существует ли перспетрон с константой, двумя входами и пороговой функцией активации, способный идеально классифицировать $y_i$ на данной выборке? А хватит ли двух таких персептронов? А может хватит трёх?
\item Введите такое преобразование исходных признаков $h_i = h(x_i, z_i)$, при котором с идеальной классификацией $y_i$ справился бы даже персептрон с одним входом, константой и пороговой функцией активации.
\end{enumerate}


\begin{sol}
\end{sol}
\end{problem}





\begin{problem}
Бандерлог из Лога\footnote{деревня в Кадуйском районе Вологодской области} ведёт блог, любит считать логарифмы и оценивать логистические регрессии. С помощью нового алгоритма Бандерлог решил задачу классификации по трём наблюдениям и получил $b_i = \hat\P(y_i = 1|x_i)$.

\begin{tabular}{cc}
  \toprule
  $y_i$ & $b_i$ \\
  \midrule
  1 & 0.7 \\
  -1 & 0.2 \\
  -1 & 0.3 \\
  \bottomrule
\end{tabular}

\begin{enumerate}
\item Постройте ROC-кривую.
\item Найдите площадь под ROC-кривой и индекс Джини.
\item Постройте PR-кривую (кривая точность-полнота).
\item Найдите площадь под PR-кривой.
\item Как по-английски будет «бревно»?
\end{enumerate}
\begin{sol}
\end{sol}
\end{problem}



\begin{problem}
Классификатор Бандерлога имеет вид
\[
a_i = \begin{cases}
1, \text{ если } b_i > t; \\
-1, \text{ иначе.}
\end{cases}
\]

Докажите, что площадь под ROC-кривой равна вероятности того, случайно выбранный положительный объект окажется позже случайно выбранного отрицательного объекта, если объекты ранжированы по возрастанию величины $b_i$.
\begin{sol}
\end{sol}
\end{problem}






\begin{problem}
Все средние издалека выглядят одинаково, $\text{среднее}=f^{-1}(0.5f(x_1) + 0.5f(x_2))$.  Например, у среднего арифметического $f(t)=t$, у среднего гармонического $f(t)=1/t$.

\begin{enumerate}
  \item Какая $f$ используется для среднего геометрического?
\end{enumerate}

Для измерения качества бинарной классификации Ара использует среднее арифметическое точности и полноты, Гена — среднее геометрическое, а Гарик — среднее гармоническое.

\begin{enumerate}[resume]
  \item У кого будут выходить самые «качественные» и самые «некачественные» прогнозы?
\end{enumerate}


\begin{sol}
\end{sol}
\end{problem}


\begin{problem}
Бандерлог начинает все определения со слов «это доля правильных ответов»:
\begin{enumerate}
  \item accuracy — это доля правильных ответов\ldots
  \item точность (precision) — это доля правильных ответов\ldots
  \item полнота (recall) — это доля правильных ответов\ldots
  \item TPR — это доля правильных ответов\ldots
\end{enumerate}

Закончите определения Бандерлога так, чтобы они были, хм, правильными.
\begin{sol}
\end{sol}
\end{problem}


\begin{problem}
Алгоритм бинарной классификации, придуманный Бандерлогом, выдаёт оценки вероятности $b_i = \hat\P(y_i=1 | x_i)$. Всего у Бандерлога 10000 наблюдений. Если ранжировать их по возрастанию $b_i$, то окажется что наблюдения с $y_i = 1$ занимают ровно места с  5501 по 5600.

Найдите площадь по ROC-кривой и площадь под PR-кривой.
\begin{sol}
\end{sol}
\end{problem}



\begin{problem}
Бандерлог собрал выборку из 900 муравьёв и 100 китов. Переменная $y_i$ равна $1$ для китов. Бандерлог хочет, чтобы его алгоритм классификации выдавал для каждого наблюдения число $b_i=f(x_i) \in [0;1]$, оценку вероятности того, что наблюдение является китом. В качестве признака Бандерлог использует количество глаз, не задумавшись о том, что оно равно двум и для муравьёв, и для китов.

Решите задачу минимизации эмпирической функции риска и найдите все $b_i$ для функций потерь:
\begin{enumerate}
  \item $L(y_i, b_i) = (y_i - b_i)^2$, если для муравьёв $y_i = 0$;
  \item $L(y_i, b_i) = |y_i - b_i|$, если для муравьёв $y_i = 0$;
  \item $L(y_i, b_i) = \begin{cases}
  -\log b_i, \text{ если } y_i = 1 \\
  -\log (1-b_i), \text{ иначе.}
  \end{cases}$;
  %\item $L(y_i, b_i) = \log (1 + \exp(-y \cdot b))$, если для муравьёв $y = -1$;
  \item $L(y_i, b_i) = \begin{cases}
  1/b_i, \text{ если } y_i = 1 \\
  1/(1-b_i), \text{ иначе.}
  \end{cases}$;
\end{enumerate}
\begin{sol}
\end{sol}
\end{problem}


\begin{problem}
Бандерлог утверждает, что открыл новую верхнюю границу для пороговой функции потерь, $\tilde{L}(M_i) = 1 + \frac{1}{\pi} \cdot \arctan(-x_i)$, где $M_i = y_i \cdot \langle w, x_i \rangle$. Прав ли бандерлог?
\begin{sol}
  Нет. Не выполнено $\tilde{L} \geq L$ для всех $M \in \RR$.
\end{sol}
\end{problem}


\begin{problem}
Бандерлог из Лога оценил логистическую регрессию по четырём наблюдениям и одному признаку с константой, получил $b_i = \hat\P(y_i = 1|x_i)$, но потерял последнее наблюдение:

\begin{tabular}{cc}
  \toprule
  $y_i$ & $b_i$ \\
  \midrule
  1 & 0.7 \\
  -1 & 0.2 \\
  -1 & 0.3 \\
  ? &  ? \\
  \bottomrule
\end{tabular}

\begin{enumerate}
\item Выпишите функцию потерь для задачи логистической регрессии.
\item Выпишите условие первого порядка по коэффициенту перед константой.
\item Помогите бандерлогу восстановить пропущенные значения!
\end{enumerate}

\begin{sol}
\end{sol}
\end{problem}


\begin{problem}
У Бандерлога три наблюдения, первое наблюдение — кит, остальные — муравьи. Киты кодируются $y_i = 1$, муравьи — $y_i = -1$. На этот раз Бандерлог, чтобы быть уверенным, что $x_i$ различаются, сам лично определил $x_i = i$. После этого Бандерлог оценивает логистическую регрессию с константой.

\begin{enumerate}
  \item Выпишите эмпирическую функцию риска, которую минимизирует Бандерлог;
  \item При каких оценках коэффициентов логистической регрессии эта функция достигает своего минимума?
\end{enumerate}

\begin{sol}
\end{sol}
\end{problem}

\begin{problem}
Рассмотрим целевую функцию логистической регрессии с константой
\[
Q(w) = \frac{1}{\ell} \sum L(y_i, b_i),
\]
где $b_i = 1 / (1 + \exp( -\langle w, x_i\rangle)$ и $L(y_i, b_i) = \begin{cases}
-\log b_i, \text{ если } y_i = 1 \\
-\log (1-b_i), \text{ иначе.}
\end{cases}$.

\begin{enumerate}
\item Найдите $dQ(w)$ и $d^2Q(w)$;
\item Найдите $dQ(0)$ и $d^2Q(0)$;
\item Выпишите квадратичную аппроксимацию для $Q(w)$ в окрестности $w=0$;
\item С какой задачей совпадает задача минимизации квадратичной аппроксимации?
\end{enumerate}

\begin{sol}
\end{sol}
\end{problem}



\begin{problem}
Винни-Пух знает, что мёд бывает правильный, $honey_i=1$, и неправильный, $honey_i=0$. Пчёлы также бывают правильные, $bee_i=1$, и неправильные, $bee_i=0$. По 100 своим попыткам добыть мёд Винни-Пух составил таблицу сопряженности:

\begin{tabular}{c|cc}
\toprule
 & $honey_i=1$ & $honey_i=0$ \\
\midrule
$bee_i=1$ & 12 & 36 \\
$bee_i=0$ & 32 & 20 \\
\bottomrule
\end{tabular}

Винни-Пух использует логистическую регрессию с константой для прогнозирования правильности мёда с помощью правильности пчёл. 

\begin{enumerate}
\item Какие оценки коэффициентов получит Винни-Пух?
\item Какой прогноз вероятности правильности мёда при встрече с неправильными пчёлами даёт логистическая модель? Как это число можно посчитать без рассчитывания коэффициентов? 
\end{enumerate}

\begin{sol}
\end{sol}
\end{problem}


\begin{problem}
Винни-Пух оценил логистическую регрессию для прогнозирования правильности мёда от высоты дерева (м) $x_i$ и удалённости от дома (км) $z_i$: $\ln odds_i = 2+0.3x_i - 0.5z_i$.
\begin{enumerate}
\item Оцените вероятность того, что $y_i=1$ для $x=15$, $z=3.5$.
\item Оцените предельный эффект увеличения $x$ на единицу на вероятность того, что $y_i=1$ для $x=15$, $z=3.5$.
\item При каком значении $x$ предельный эффект увеличения $x$ на единицу в точке $z=3.5$ будет максимальным?
\end{enumerate}

\begin{sol}
Предельный эффект максимален он при максимальной производной $\Lambda'(\hat \beta_1 + \hat\beta_2x + \hat\beta_3z)$, то есть при $\hat \beta_1 + \hat\beta_2x + \hat\beta_3z=0$.
\end{sol}
\end{problem}

% \section{Хочу ещё задач!}




\Closesolutionfile{solution_file}


% для гиперссылок на условия
% http://tex.stackexchange.com/questions/45415
\renewenvironment{solution}[1]{%
         % add some glue
         \vskip .5cm plus 2cm minus 0.1cm%
         {\bfseries \hyperlink{problem:#1}{#1.}}%
}%
{%
}%

%\section{Решения}
%\input{all_solutions}

%\section{Источники мудрости}
%\printbibliography[heading=none]


\end{document}
